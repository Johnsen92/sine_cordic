\section{Implementation}
We decided to use a pipeline design with generic pipeline length because the software algorithm strongly suggested it. Furthermore we used a modular design to keep the complexity low and to leave room for later improvements.

Our design structure contains a top entity called \textit{sine\_cordic} in which the wiring of the subcomponents and the synchronization of the pipeline stages takes place.

Our subentities and at the same time our pipeline stages consist of an initialization stage called \textit{cordic\_init}, an entity that represents a single iteration of the algorithm called \textit{cordic\_step} and a final muliplication entity simply called \textit{mult} to execute the multiplication with a correction factor as final step of the algorithm. 

\subsection{angles and Kvalues}

For the values of \textit{angles} and \textit{Kvalues} we chose to implement them as functions
instead of a predefined array like it is done in the reference algorithm. This provides the possibility to define an arbitrarily large number of iterations without dropping accuracy from running out of array elements (like it happens in the reference algorithm).

\subsection{cordic\_init.vhd}

This entity is used to align the input angle to a value between \(-\pi/2\) and \(\pi/2\). Its functionality corresponds to the first part of the reference algorithm:

\begin{lstlisting}[language=Matlab]

if beta < -pi/2 || beta > pi/2
if beta < 0
v = cordic(beta + pi, n);
else
v = cordic(beta - pi, n);
end
v = -v; % flip the sign for second or third quadrant
return
end

\end{lstlisting}

It contains the following ports and generics: 

\begin{center}
	\begin{tabular}{ | l | c | l | }
		\hline
		\textbf{Generic} & \textbf{Default} & \textbf{Description} \\
		\hline
		DATA\_WIDTH & 8 & Defines the bit width of the signales in this entity \\
		\hline
	\end{tabular} 
\end{center}

\begin{center}
	\begin{tabular}{ | l | c | l | }
		\hline
		\textbf{Portname} & \textbf{Direction} & \textbf{Description} \\
		\hline
		beta\_in & in & Input beta angle value \\
		sine\_in & in  & Initial sine value \\
		cosine\_in & in  & Initial cosine value \\
		beta\_out & out  & Beta angle value aligned to be in the range \(-\pi/2\) to \(\pi/2\) \\
		sine\_out & out  & Initial sine value with aligned sign \\
		cosine\_out & out  & Initial sine value with aligned sign \\
		\hline
	\end{tabular} 
\end{center}

While in the reference algorithm the input angle can be arbitrarily large, in our case the angle is bound by the fixpoint comma format of our input port which, at this state, consists of 3 integer bits including 1 sign bit and genericly many fractional bits. We chose the integer part to be 3 bits long because it covers the range from \(-\pi\) to \(\pi\) which is sufficient to calculate every possible sine value of one full period. As the maximum input angle in our case is approximately 4 (depending on the number of fraction bits) and the minimum angle is -4, only up to one iteration of this code section is needed.


\subsection{cordic\_step.vhd}

The \textit{cordic\_step} entity implements one single iteration of the algorithm as described in the software example. The entity has five input ports and three output ports:

\begin{center}
	\begin{tabular}{ | l | c | l | }
		\hline
		\textbf{Generic} & \textbf{Default} & \textbf{Description} \\
		\hline
		DATA\_WIDTH & 8 & Defines the bit width of the signals in this entity \\
		\hline
	\end{tabular} 
\end{center}

\begin{center}
	\begin{tabular}{ | l | c | l | }
		\hline
		\textbf{Portname} & \textbf{Direction} & \textbf{Description} \\
		\hline
		poweroftwo & in & Defines the factor to be used in the matrix multiplication \\
		alpha & in  & The precalculated angle to be added to the angle beta \\
		beta\_in & in  & The current beta value given as input for this iteration \\
		sine\_in & in  & The current sine value given as input for this iteration \\
		cosine\_in & in  & The current cosine value given as input for this iteration \\
		beta\_out & out  & The altered beta value calculated by this instance of the component \\
		sine\_out & out  & The altered sine value calculated by this instance of the component \\
		cosine\_out & out  & The altered beta cosine calculated by this instance of the component \\
		\hline
	\end{tabular} 
\end{center}

One instance of the component represents one iteration of the algorithm shown in section 1.1 and performs exactly the same operations. In the loop of the refernce algorithm \textit{poweroftwo} refers to a variable with the initial value 1 which is subsequently devided by two in each iteration (representing the term \(2^{-n}\) where \textit{n} is the loop variable) and used in the matrix multiplication to determine the new values of the sine and cosine vector. In our case the same functionality is implemented by simply shifting the current value for \textit{n} bits to the right which is matematically the same operation.

Instead of calculating a \textit{sigma} to determine the sign of the factor used in the matrix multiplication as it is done in the reference algorithm, we decided to simply use an if-statement checking the sign of \textit{beta} and perform either addition or subtraction accordingly. By doing so we completely eradicated the matrixmultiplication from our implementation and broke it down to a sequence of shifts, checks, additions and subtractions. This was to keep the implementation simple and to maintain good performance due to multiplication in VHDL being much slower than the operations mentioned above.

\subsection{mult.vhd} 

The \textit{mult} entity is used for the final step of the algorithm: The multiplication with the correction factor \textit{Kn}. It provides the following ports and generics:

\begin{center}
	\begin{tabular}{ | l | c | l | }
		\hline
		\textbf{Generic} & \textbf{Default} & \textbf{Description} \\
		\hline
		DATA\_WIDTH & 8 & Defines the bit width of the signals in this entity \\
		\hline
	\end{tabular} 
\end{center}

\begin{center}
	\begin{tabular}{ | l | c | l | }
		\hline
		\textbf{Portname} & \textbf{Direction} & \textbf{Description} \\
		\hline
		clk & in & Clock signal \\
		reset & in  & Reset signal \\
		dataa & in  & First factor of the multiplication \\
		datab & in  & Second factor of the multiplication \\
		result & out  & Result of the multiplication \\
		\hline
	\end{tabular} 
\end{center}

The bit width of the result signal is of course double the length of dataa and datab.

\subsection{sine\_cordic.vhd}

This entity implements the wiring and synchronization of the sub entities. It is designed to take and produce input and output of variable precision respectively. The precision can be adjusted via the generic variables of the entity:

\begin{center}
	\begin{tabular}{ | l | c | l | }
		\hline
		\textbf{Generic} & \textbf{Default} & \textbf{Description} \\
		\hline
		INPUT\_DATA\_WIDTH & 8 & Defines the bit width of the input data signal \\
		OUTPUT\_DATA\_WIDTH & 8 & Defines the bit width of the internal and output data signals \\
		ITERATION\_COUNT & 12 & Defines the number of loop iterations \\
		\hline
	\end{tabular} 
\end{center}

The entity comes with the following ports:

\begin{center}
	\begin{tabular}{ | l | c | l | }
		\hline
		\textbf{Portname} & \textbf{Direction} & \textbf{Description} \\
		\hline
		reset & in & Reset signal \\
		clk & in  & Clock signal \\
		beta & in  & Input angle for the sine function \\
		start & in & Control flag that states the \textit{beta} port holds valid data \\
		done & out & Control flag that states the \textit{result} port holds valid data \\
		result & out & Output port for the result of the sine function \\
		\hline
	\end{tabular} 
\end{center}

The generic \textit{ITERATION\_COUNT} defines how many instances of the \textit{cordic\_step} entity are put together in the pipeline. 
