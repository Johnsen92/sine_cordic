\section{Introduction}
Our task was to implement an algorithm that performs a sine calculation for a given angle using the CORDIC ( \textbf{CO}ordinate \textbf{R}otation \textbf{DI}gital \textbf{C}omputer) algorithm. We used the Matlab software example code from Wikipedia [\href{https://en.wikipedia.org/wiki/CORDIC}{https://en.wikipedia.org/wiki/CORDIC}] as a guideline for our implementation. For developement and simulation we used ModelSim.


\subsection{Reference algorithm implemented in Matlab}

\begin{lstlisting}[language=Matlab]
function v = cordic(beta,n)
% This function computes v = [cos(beta), sin(beta)] (beta in radians)
% using n iterations. Increasing n will increase the precision.

if beta < -pi/2 || beta > pi/2
	if beta < 0
		v = cordic(beta + pi, n);
	else
		v = cordic(beta - pi, n);
	end
	v = -v; % flip the sign for second or third quadrant
	return
end

% Initialization of tables of constants used by CORDIC
% need a table of arctangents of negative powers of two, in radians:
% angles = atan(2.^-(0:27));
angles =  [  ...
0.78539816339745   0.46364760900081   0.24497866312686   0.12435499454676 ...
0.06241880999596   0.03123983343027   0.01562372862048   0.00781234106010 ...
0.00390623013197   0.00195312251648   0.00097656218956   0.00048828121119 ...
0.00024414062015   0.00012207031189   0.00006103515617   0.00003051757812 ...
0.00001525878906   0.00000762939453   0.00000381469727   0.00000190734863 ...
0.00000095367432   0.00000047683716   0.00000023841858   0.00000011920929 ...
0.00000005960464   0.00000002980232   0.00000001490116   0.00000000745058 ];
% and a table of products of reciprocal lengths of vectors [1, 2^-2j]:
% Kvalues = cumprod(1./abs(1 + 1j*2.^(-(0:23))))
Kvalues = [ ...
0.70710678118655   0.63245553203368   0.61357199107790   0.60883391251775 ...
0.60764825625617   0.60735177014130   0.60727764409353   0.60725911229889 ...
0.60725447933256   0.60725332108988   0.60725303152913   0.60725295913894 ...
0.60725294104140   0.60725293651701   0.60725293538591   0.60725293510314 ...
0.60725293503245   0.60725293501477   0.60725293501035   0.60725293500925 ...
0.60725293500897   0.60725293500890   0.60725293500889   0.60725293500888 ];
Kn = Kvalues(min(n, length(Kvalues)));

% Initialize loop variables:
v = [1;0]; % start with 2-vector cosine and sine of zero
poweroftwo = 1;
angle = angles(1);

% Iterations
for j = 0:n-1;
	if beta < 0
		sigma = -1;
	else
		sigma = 1;
	end
	factor = sigma * poweroftwo;
	% Note the matrix multiplication can be done using scaling by powers of two and addition subtraction
	R = [1, -factor; factor, 1];
	v = R * v; % 2-by-2 matrix multiply
	beta = beta - sigma * angle; % update the remaining angle
	poweroftwo = poweroftwo / 2;
	% update the angle from table, or eventually by just dividing by two
	if j+2 > length(angles)
		angle = angle / 2;
	else
		angle = angles(j+2);
	end
end

% Adjust length of output vector to be [cos(beta), sin(beta)]:
v = v * Kn;
return

endfunction
\end{lstlisting}

